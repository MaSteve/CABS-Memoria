\chapter{Traducción a ABS}
La traducción de CABS a ABS forma el grueso de este trabajo. La diferencia de paradigmas entre un lenguaje y otro hace que sea necesario la implementación de algunas estructuras de datos adicionales ausentes en ABS.\\

Además de dichas estructuras, es necesario emplear algunos trucos para poder vencer las restricciones del lenguaje para crear el concepto de variable global o de llamada a funciones síncrona.\\

En este capítulo discutiremos este tema de una forma abstracta para posteriormente poder llevar a cabo la implementación de un compilador correcto.

\section{Variables globales y funciones}
La ausencia de memoria compartida entre los distintos \emph{COGs} (Concurrent Object Group) o conjunto de tareas de un objeto hace que la idea de variable global no sea inmediata. Del mismo modo, es necesario discutir el concepto de función al estilo de C, pese a que los métodos de una interfaz en ABS sean públicos y a primera vista similares.\\

Una primera aproximación vendría dada por el uso de una única clase en la que encapsular todo nuestro programa. Esta clase implementaría una interfaz con todas las cabeceras de las funciones de nuestro programa. Además contaría entre sus atributos con las variables globales, consiguiendo de este modo una visibilidad completa desde cualquier punto del programa.\\

Veamos qué ocurre con el siguiente código de ejemplo en CABS. En él se puede ver como se llama a la función $f$ con $\thread$ haciendo que las dos asignaciones de la variable $var1$ puedan entrelazarse.\\

\lstinputlisting{example.txt}

En otras palabras, queremos que nuestra traducción a ABS pueda llegar a ambos \emph{interleavings}, es decir, que el valor final de $var1$ pueda ser $1$ o $2$.\\

Empleando la primera aproximación, obtendríamos un código similar al siguiente.

\lstinputlisting{code.2.1.txt}

Sin embargo, esta traducción nos lleva a un código ABS donde solo es posible uno de los dos \emph{interleavings} (concretamente el que termina con $var1$ valiendo $2$).\\

El motivo por el que ocurre esto es por la cola de tareas del objeto que hemos discutido anteriormente. En este programa existe un único objeto que se corresponde con la instancia $prog$. Cuando la función $main$ llama asíncronamente a $f$ se realiza un paso de mensaje al mismo objeto, que lo almacena en una cola a la espera de poder ejecutarlo, es decir, a que termine la ejecución del mensaje actual que se corresponde con la llamada a $main$.\\

Un intento para solucionar esta situación podría pasar por usar un $suspend$ tras llamar a $f$. Esto forzaría la deselección de la tarea actual y permitiría escoger entre retomarla o empezar con el otro mensaje que se encuentra en la cola.

\begin{lstlisting}
  this!f();
  suspend;
\end{lstlisting}

De este modo conseguiremos que el mensaje del $main$ pueda dejar paso a la llamada a $f$, permitiendo los dos \emph{interleavings}.\\

Pero, ¿qué ocurre si los cuerpos de $f$ y $main$ son un poco más largos como en el siguiente ejemplo?

\lstinputlisting{code.2.2.txt}

En este caso no hay forma de regresar a la función $main$ antes de que termine la ejecución de $f$ suponiendo que se escoja esta tras el $suspend$. De nuevo solo conseguimos la mitad de los 4 estados finales posibles.\\

Como emplear un $suspend$ detrás de cada instrucción haría difícil de entender el código resultante, la única alternativa que nos queda es tener cada función en un objeto distinto y realizar una instanciación cada vez que queramos realizar una llamada.\\

Esta situación resuelve los problemas anteriormente planteados porque cada cola de tareas de un objeto actúa como si se ejecutara en un procesador independiente. Queda ahora resolver cómo conseguir que haya variables compartidas entre los distintos COGs.\\

La solución pasa por crear un objeto independiente en el que se almacenen las variables globales como atributos, a los que solo se pueda acceder a través de unos métodos \emph{getters} y \emph{setters}. Este objeto sería el primero en crearse y se pasaría como parámetro en la creación de los sucesivos como un parámetro de clase.\\

Las llamadas a los mencionados \emph{getters} y \emph{setters} podrían hacerse usando una llamada asíncrona e inmediatamente haciendo un $await$ de esta. Este $await$ sería necesario tanto para lecturas (como se vio en la discusión de los tipos futuros en el capítulo anterior) como en las escrituras, puesto que un mismo objeto podría intentar hacer dos escrituras seguidas sobre la misma variable y el orden de ellas debe mantenerse. Esto último se puede ver en el siguiente ejemplo:
\begin{lstlisting}
  globalval!setvar(1);
  globalval!setvar(2);
\end{lstlisting}

Si no realizamos un $await$ en las llamadas asíncronas, podríamos tener la situación en la que el \emph{scheduler} invirtiera el orden efectivo de ejecución de las llamadas, lo que daría un comportamiento no predecible en una situación en la que se esperaría siempre obtener el mismo resultado.\\

Usando esta idea para las variables globales, el último ejemplo se podría traducir al siguiente código ABS:

\lstinputlisting{code.2.3.txt}

Analicemos por qué esto funciona. La ejecución de $f$ y del resto de la función $main$ se ejecutan ahora en paralelo. Ambas funciones tienen acceso al objeto $globalval$ por ser un parámetro de clase. Cuando $f$ o $main$ realizan una escritura sobre $var1$ llaman al método setter correspondiente. Puede ocurrir que uno de los dos objetos realice la llamada y obtenga el valor de retorno antes que el otro llame al método o que los dos llamen a la vez (relativamente) y hagan un await.\\

En este segundo caso hay que analizar que no haya problemas de deadlock. Los métodos del objeto $globalval$ no realizan ninguna llamada a ninguna otra función. Solamente cogen el valor solicitado y lo devuelven, en el caso de las lecturas, o escriben un valor en un atributo, en el caso de las escrituras. Cuando dos o más llamadas a un método ocurren a la vez estas se introducen en la cola de mensajes del objeto y se gestionan arbitrariamente. De este modo se garantiza que las llamadas a $globalval$ terminan en algún momento y la arbitrariedad en la gestión de las llamadas garantiza la existencia de todos los interleavings posibles sobre la variable $var1$.\\

Del mismo modo se procede con la variable $var2$. Además, gracias al $await$, dentro de un mismo método, como en la función $f$, se garantiza que no hay un desorden entre las asignaciones, en el caso de $f$, las asignaciones a $var1$ y a $var2$. De forma análoga esto funcionaría con un ejemplo en el que se hicieran lecturas a una variable global.\\

Con esto hemos conseguido solucionar el problema de las variables globales y a la vez hemos obtenido una concurrencia de grano fino entre las distintas hebras de CABS al separar las lecturas de las escrituras en dos llamadas asíncronas distintas.

\section{Arrays locales y globales}

ABS cuenta un tipo genérico de lista recursiva de coste de acceso lineal. Se trata de un tipo inmutable que no permite la modificación del contenido de la lista una vez creada. Pese a tener estas restricciones podemos conseguir un comportamiento similar al de un array común obviando el coste de las operaciones.\\

Para ello nos interesará encapsular el tipo lista en una clase ABS que implemente una interfaz con dos métodos que aporten el concepto de array:
un \emph{getter} y un \emph{setter} de elementos por posición.\\

La implementación de un array de enteros quedaría del siguiente modo:

\lstinputlisting{code.2.4.txt}

La implementación del \emph{setter} pasa por iterar sobre la lista hasta encontrar la posición solicitada para posteriormente concatenar dos listas con el elemento modificado en medio. El \emph{getter} por otro lado emplea una llamada a una función de ABS que obtiene el elemento n-ésimo de un lista. Una implementación similar se puede hacer para el otro tipo de nuestro lenguaje CABS: los booleanos.\\

El hecho de tener ahora nuestro tipo array encapsulado en una clase ABS nos permite utilizarlo con facilidad para implementar los arrays globales. Para este propósito podemos crear un método $init$ que se encargue de la creación de los objetos array y unos métodos \emph{getters} y \emph{setters} modificados que se encarguen de realizar las llamadas a los métodos finales del array. La razón del método $init$ es porque ABS no permite la creación de objetos al declarar los atributos de una clase, cosa que sí permite con los tipos primitivos.\\

El código generado para el siguiente ejemplo

\lstinputlisting{code.2.5.txt}

quedaría de este modo:

\lstinputlisting{code.2.6.txt}

Por último, sería conveniente (y lo será más adelante en la traducción de las funciones) tener un método que nos devuelva el array global para usarlo con algunos fines locales (concretamente el paso de arrays por referencia en los argumentos de una función). Esto se consigue con un método $retrieve$ que implemente la clase $GlobalVariables$. Para el ejemplo anterior, quedaría como

\lstinputlisting{code.2.7.txt}

No hay que decir que las clases que hemos implementado en esta sección permiten su uso en cualquier ámbito local de un método, con lo que también contaremos con arrays locales en ABS.

\subsection{Arrays multidimensionales}

El lenguaje CABS permite en su sintaxis declarar arrays multidimensionales al estilo de C. ABS no cuenta con un tipo de datos similar, pero, gracias a la propuesta previamente expuesta, podemos simular el comportamiento de las matrices estableciendo una biyección con la representación de un array de tamaño el producto de los tamaños de las dimensiones de la matriz. En otras palabras, la traducción implementada en este trabajo considera las matrices como un array unidimensional.\\

A modo de explicación, sea $mat$ una matriz multidimensional entera en $int^{N_1} \times \dots \times int^{N_D}$ y supongamos que queremos acceder a la posición $(i_1, \dots, i_D)$ donde $\forall j \in \{1, \dots, D\}$ tenemos $i_j \in \{0, \dots, N_j - 1\}$, entonces la posición $p$ correspondiente a dicho elemento en un array en $int^{\prod_{i = 1}^{D} N_i}$ vendría dada por $p = \sum_{l = 1}^{D} i_l \cdot (\prod_{j=1}^{l - 1} N_j)$.

\section{Expresiones aritméticas}

CABS presenta una gran flexibilidad en sus expresiones aritmético-lógicas no presentes en ABS. La sintaxis de ABS obliga a que el valor de retorno devuelto por una función solo pueda ser asignado a una variable, no pudiendo ser usado inmediatamente en una expresión del mismo tipo que el de retorno. Esto nos obliga a traducir las expresiones aritméticas en una serie de asignaciones auxiliares cuyas variables posteriormente se operan con los valores almacenados en ellas. Será por tanto necesario tener un modo de obtener nombres de variables auxiliares que no se referencien en ningún otro punto del programa traducido.\\

Sea por tanto $\Aux: \Nat \rightarrow \Var$ una función definida sobre los enteros que devuelve el n-ésimo nombre de variable {\it libre}. En un sentido estricto esta función debería tomar como argumento el programa en CABS y la traducción parcial en ABS para saber qué nombres de variable han sido ya usados. A modo de simplificación podemos evitar esto reservando un conjunto de nombres de variables para este propósito que no sean accesibles al usuario. Esta es la idea que posteriormente se llevará a cabo en la implementación del compilador. De ahora en adelante nos referiremos a este conjunto como $\Var_R \subset \Var$. Un posible ejemplo de subconjunto $\Var_R$ podría ser $\{aux\_var\_v: v \in \Nat\}$, que por tener la misma cardinalidad que $\Nat$ nos permite crear una biyección inmediata.\\

Teniendo esto en cuenta, resulta fácil pensar en definir la traducción de las expresiones como una función que toma una expresión en CABS junto con un natural $v$ y que devuelve una expresión en ABS junto con un natural que indique el siguiente natural no utilizado en la traducción. Si garantizamos no repetir un mismo $n$ para dos expresiones distintas entonces garantizamos que las variables auxiliares no son usadas más allá de una única asignación y una única referencia.\\

Especificamos esta idea con la definición de las funciones $\C_\AExp: \AExp \rightarrow \Nat \rightarrow (\ABS \times \Nat)$ y su homóloga  $\C_\BExp$ para las expresiones booleanas:
\begin{align*}
  \C_\AExp\eval{n} v &= (Int\, (\Aux\, v) = n, v + 1) \\
  \C_\AExp\eval{x} v &= (Int\, (\Aux\, v) = x, v + 1) \mbox{ donde $x$ es variable entera local.} \\
  \C_\AExp\eval{x} v &= (Int\, (\Aux\, v) = await \mbox{ globalval!get}x(), v + 1) \\
  &\mbox{ donde $x$ es variable entera global.} \\
  \C_\AExp\eval{a_1 \odot_\A a_2} v &= (c_1\; c_2 \; Int\, (\Aux\, v'') =  (\Aux\, (v' - 1))  \odot_\A  (\Aux\, (v'' - 1)) , v'' + 1) \\
  &\mbox{donde $\C_\AExp\eval{a_1} v = (c_1, v')$ y $\C_\AExp\eval{a_2}v' = (c_2, v'')$} \\
  \C_\BExp\eval{false} v &= (Bool\, (\Aux\, v) = False, v + 1) \\
  \C_\BExp\eval{true} v &= (Bool\, (\Aux\, v) = True, v + 1) \\
  \C_\BExp\eval{x} v &= (Bool\, (\Aux\, v) = x, v + 1) \mbox{ donde $x$ es variable booleana local.} \\
  \C_\BExp\eval{x} v &= (Bool\, (\Aux\, v) = await \mbox{ globalval!get}x(), v + 1) \\
  &\mbox{ donde $x$ es variable booleana global.} \\
  \C_\BExp\eval{b_1 \odot_\Bb b_2} v &= (c_1\; c_2 \; Bool\, (\Aux\, v'') =  (\Aux\, (v' - 1))  \odot_\Bb  (\Aux\, (v'' - 1)) , v'' + 1) \\
  &\mbox{donde $\C_\BExp\eval{b_1} v = (c_1, v')$ y $\C_\BExp\eval{b_2}v' = (c_2, v'')$} \\
  \C_\BExp\eval{a_1 \odot a_2} v &= (c_1\; c_2 \; Bool\, (\Aux\, v'') =  (\Aux\, (v' - 1))  \odot  (\Aux\, (v'' - 1)) , v'' + 1) \\
  &\mbox{donde $\C_\AExp\eval{a_1} v = (c_1, v')$ y $\C_\AExp\eval{a_2}v' = (c_2, v'')$ y $\odot$ comparador.}
\end{align*}
Para las llamadas a funciones supondremos por el momento que las interfaces y clases asociadas se encuentran traducidas en algún otro punto del código y que por tanto podemos hacer uso de sus nombres.

\begin{align*}
  \C_\AExp\eval{f(e_1 \dots e_n)} v^{1} &= (c_1 \dots c_n\\
  &\mbox{Int$f\,(\Aux\, (v^{n+1})) =$ new Imp$f($globalval$)$} \\
  &Int\, (\Aux\, (v^{n+1} + 1)) = (\Aux\, (v^{n+1}))!((\Aux\, (v^{2} - 1)), \dots,\\
  &(\Aux\, (v^{n+1} - 1))), v^{n+1} + 2) \\
  &\mbox{donde $\C_\Exp\eval{e_i} v^{i} = (c_i, v^{i+1})$ (escogiendo $\C_\AExp$ o $\C_\BExp$ según convenga)} \\
  \C_\BExp\eval{f(e_1 \dots e_n)} v^{1} &= (c_1 \dots c_n\\
  &\mbox{Int$f\,(\Aux\, (v^{n+1})) =$ new Imp$f($globalval$)$} \\
  &Bool\, (\Aux\, (v^{n+1} + 1)) = (\Aux\, (v^{n+1}))!((\Aux\, (v^{2} - 1)), \dots,\\
  &(\Aux\, (v^{n+1} - 1))), v^{n+1} + 2) \\
  &\mbox{donde $\C_\Exp\eval{e_i} v^{i} = (c_i, v^{i+1})$ (escogiendo $\C_\AExp$ o $\C_\BExp$ según convenga)} \\
\end{align*}

\section{Traducción de código}

Tras esta introducción, podemos vislumbrar qué aspecto tendrá la traducción de código llevada a cabo en este trabajo. Obviaremos el preámbulo de los programas ABS, que incluirá la definición de los clases Array, y nos centraremos en otros aspectos más importantes.\\

Sea pues la función $\C: \CABS \rightarrow \Nat \rightarrow (ABS \times \Nat)$ nuestra función de traducción de CABS a ABS definida recursivamente del siguiente modo:

\begin{align*}
  \C\eval{int\, var\;} v &= (Int\, var, v) \\
  \C\eval{bool\, var\;} v &= (Bool\, var, v) \\
  \C\eval{var = a\;} v &= (c\; var = (\Aux\, (v' - 1)), v') \\
  & \mbox{donde $\C_\AExp\eval{a} v = (c, v')$ y $var$ variable local entera} \\
  \C\eval{var = b\;} v &= (c\; var = (\Aux\, (v' - 1)), v') \\
  & \mbox{donde $\C_\BExp\eval{b} v = (c, v')$ y $var$ variable local booleana} \\ %%
  \C\eval{var = a\;} v &= (c\; await \mbox{ globalval!set}var(\Aux\, (v' - 1)), v') \\
  & \mbox{donde $\C_\AExp\eval{a} v = (c, v')$ y $var$ variable global entera} \\
  \C\eval{var = b\;} v &= (c\; await \mbox{ globalval!set}var(\Aux\, (v' - 1)), v') \\
  & \mbox{donde $\C_\BExp\eval{b} v = (c, v')$ y $var$ variable global booleana} \\ %%
  \C\eval{S_1 S_2} v &= (c_1 c_2, v'') \mbox{ donde $\C\eval{S_1} v = (c_1, v')$ y $\C\eval{S_2} v' = (c_2, v'')$} \\
  \C\eval{\If{b}{S_1}{S_2}} v &= (c_1\; \If{\Aux\, (v' - 1)}{c_2}{c_3}, v''') \\
  & \mbox{ donde $\C_\BExp\eval{b} v = (c_1, v')$, $\C\eval{S_1} v' = (c_2, v'')$ y $\C\eval{S_2} v'' = (c_3, v''')$} \\
  \C\eval{\while{b}{S}} v &= (c_1\; \while{\Aux\, (v' - 1)}{c_2 c_3\, \Aux\, (v' - 1) = \Aux\, (v''' - 1)\;}, v''') \\
  & \mbox{ donde $\C_\BExp\eval{b} v = (c_1, v')$, $\C\eval{S} v' = (c_2, v'')$ y $\C_\BExp\eval{b} v'' = (c_3, v''')$}
\end{align*}
Sobre la traducción del while cabe destacar que es necesario traducir dos veces su condición y el uso de una asignación adicional. Esto es debido a que el concepto de while obliga a evaluar su condición antes de la ejecución de su cuerpo en cada una de las iteraciones para decidir si el salto se toma o no. Por el modo en que se han traducido las expresiones aritmético-lógicas, la condición en ABS de un while se reduce a comprobar el valor asignado a una variable auxiliar y es por esto por lo que, sobre dicha variable, se hará una asignación adicional al final de toda iteración.\\

En segundo lugar, hace falta mencionar que las funciones de traducción de expresiones booleanas no contemplan la reutilización de variables, por lo que aunque la condición sea la misma, se emplearán unos nombres de variables nuevos al final del cuerpo del while respecto a los que se usaban al evaluar la condición fuera del cuerpo. A efectos prácticos, es fácil demostrar que los nombres de las variables auxiliares no influyen en el cómputo de la expresión booleana. De hecho se ve claramente que la traducción es la misma salvo un offset en el natural que identifica a las variables auxiliares.\\

Es por esto por lo que podemos entender que en la traducción del código se puede indistintamente sustituir el código $c_3$ por $c_2$ en la regla del while si por motivos de simplicidad hiciera falta a la hora de desarrollar la corrección de esta traducción. En una implementación real esto no es inmediato porque los compiladores o interpretes de ABS no permiten la declaración duplicada de variables, problema que en este trabajo no nos atañe a nivel abstracto.\\

Para la traducción de las funciones necesitamos previamente una traducción de los argumentos $\C_{arg}: \Args \rightarrow \Args_\ABS$ que en esencia lo único que hace es cambiar los tipos de CABS a los de ABS.
\begin{align*}
  \C_{arg}\,\varepsilon &= \varepsilon\\
  \C_{arg}\, (int \, var,\,arg) &= Int\,var,\,\C_{arg}\, arg\\
  \C_{arg}\, (bool \, var,\,arg) &= Bool\,var,\,\C_{arg}\, arg
\end{align*}

Usando esta traducción de argumentos tenemos que
\begin{align*}
  \C\eval{int\,\func(arg)\{S\,\return a\}} v &= (\mbox{interface Int$\func$}\{
  \mbox{$Int\, \func(\C_{arg}\,arg)\;$}\}\\
  & \mbox{class Imp$\func($GLOBAL globalval$)$ implements Int$\func\{$}\\
  & Int\, func(\C_{arg}\,arg) \{ c\,c'\,\return \Aux\, (v'' - 1)\}\} , v'')\\
  & \mbox{donde $\C\eval{S} v = (c, v')$ y } \C_\AExp\eval{a} v' = (c', v'')\\
  \C\eval{bool\,\func(arg)\{S\,\return b\}} v &= (\mbox{interface Int$\func$}\{
  \mbox{$Bool\, \func(\C_{arg}\,arg)\;$}\}\\
  & \mbox{class Imp$\func($GLOBAL globalval$)$ implements Int$\func\{$}\\
  & Bool\, func(\C_{arg}\,arg) \{ c\,c'\,\return \Aux\, (v'' - 1)\}\} , v'')\\
  & \mbox{donde $\C\eval{S} v = (c, v')$ y } \C_\BExp\eval{b} v' = (c', v'')\\
  \C\eval{void\,\func(arg)\{S\}} v &= (\mbox{interface Int$\func$}\{
  \mbox{$Unit\, \func(\C_{arg}\,arg)\;$}\}\\
  & \mbox{class Imp$\func($GLOBAL globalval$)$ implements Int$\func\{$}\\
  & Unit\, func(\C_{arg}\,arg) \{ c \}\} , v')
  \mbox{ donde $\C\eval{S} v = (c, v')$}
\end{align*}

y para las variables globales tenemos
\begin{align*} % await globalval!setvar1(1);
  \C\eval{var = a\;} v &= (c\; await \mbox{ globalval!set}var(\Aux\, (v' - 1)), v') \\
  & \mbox{donde $\C_\AExp\eval{a} v = (c, v')$ y $var$ variable global entera} \\
  \C\eval{var = b\;} v &= (c\; await \mbox{ globalval!set}var(\Aux\, (v' - 1)), v') \\
  & \mbox{donde $\C_\BExp\eval{b} v = (c, v')$ y $var$ variable global booleana}
\end{align*}\\

Por último, hablaremos de la traducción del $\thread$. Como ya hemos comentado en las secciones anteriores, la traducción propuesta es similar a la de las llamadas a función con la diferencia de que no se espera a la resolución del valor futuro de retorno. Su regla es

\begin{align*}
\C\eval{\thread f(e_1 \dots e_n)} v^{1} &= (c_1 \dots c_n\\
  &\mbox{Int$f\,(\Aux\, (v^{n+1})) =$ new Imp$f($globalval$)$} \\
  &(\Aux\, (v^{n+1}))!((\Aux\, (v^{2} - 1)), \dots, (\Aux\, (v^{n+1} - 1))), v^{n+1} + 1) \\
  &\mbox{donde $\C_\Exp\eval{e_i} v^{i} = (c_i, v^{i+1})$ (escogiendo $\C_\AExp$ o $\C_\BExp$ según convenga)} \\
\end{align*}

\subsection{Traducción de arrays}

Como comentamos ya en el capítulo sobre la semántica de CABS, los arrays quedarán excluidos de la demostración formal y es por ello por lo que hemos decidido no incluirlos en la definición de la traducción. No obstante, la implementación sí los tiene en cuenta y lleva a cabo la compilación teniendo en cuenta lo comentado en la sección sobre arrays de este capítulo.
