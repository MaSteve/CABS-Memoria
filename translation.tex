\chapter{Traducción a ABS}
La traducción de CABS a ABS forma el grueso de este trabajo. La diferencia de paradigmas entre un lenguaje y otro hace que sea necesario la implementación de algunas estructuras de datos adicionales ausentes en ABS.\\

Además de dichas estructuras, es necesario emplear algunos trucos para poder vencer las restricciones del lenguaje para crear el concepto de variable global o de llamada a funciones síncrona.\\

En este capítulo discutiremos este tema de una forma abstracta para posteriomente poder llevar a cabo la implementación de un compilador correcto.

\section{Variables globales y funciones}
La ausencia de memoría compartida entre los distintos cogs hace que la idea de variable global no sea inmediata. Del mismo modo, es necesario discutir el concepto de función al estilo de C, pese a que los métodos de una interfaz en ABS sean públicos y a primera vista similares.\\

Una primera aproximación vendría dada por el uso de una única clase en la que encapsular todo nuestro programa. Esta clase implementaría una interfaz con todas las cabeceras de las funciones de nuestro programa. Además contaría entre sus atributos las variables globales, consiguiendo de este modo una visibilidad completa desde cualquier punto del programa.\\

Veamos que ocurre con el siguiente código de ejemplo en CABS. En él se puede ver como se llama a la función $f$ con $\thread$ haciendo que las dos asignaciones de la variable $var1$ puedan entrelazarse.\\

\lstinputlisting{example.txt}

En otras palabras, queremos que nuestra traducción a ABS pueda llegar a ambos interleavings, es decir, que el valor final de $var1$ pueda ser $1$ o $2$.\\

Empleando la primera aproximación, obtendríamos un código similar al siguiente.

\lstinputlisting{code.2.1.txt}

Sin embargo, esta traducción nos lleva a un código ABS donde solo es posible uno de los dos interleavings (concretamente el que termina con $var1$ valiendo $2$).\\

El motivo por el que ocurre esto es por el concepto de cog discutido anteriormente. (AÑADIR INTRO ABS!!!) En este programa existe un único cog que se corresponde con la instancia $prog$. Cuando la función $main$ llama asíncronamente a $f$ se realiza un paso de mensaje al mismo objeto, que lo almacena en una cola a la espera de poder ejecutarlo, es decir, a que termine la ejecución del mensaje actual que se corresponde con la llamada a $main$.\\

Un intento para solucionar esta situación podría pasar por usar un $await$ en la llamada a $f$.

\begin{lstlisting}
   await this!f();
\end{lstlisting}

De este modo conseguiremos que el mensaje del $main$ pueda dejar paso a otro mensaje de la cola como la llamada a $f$ permitiendo los dos interleavings.\\

Pero, ¿qué ocurre si el cuerpo de $f$ y $main$ son un poco más largos como en el siguiente ejemplo?

\lstinputlisting{code.2.2.txt}

En este caso no hay forma de regresar a la función $main$ antes de que termine la ejecución de $f$ suponiendo que se escoja tras el await. De nuevo solo conseguimos la mitad de los 4 estados finales posibles.\\

La única alternativa que nos queda es tener cada función en un cog distinto. Y realizar una instanciación cada vez que realicemos una llamada.\\

Esta situación resuelve los problemas anteriormente planteados porque cada cog actúa como si se ejecutara en un procesador independiente. Queda ahora resolver cómo conseguir que haya variables compartidas entre los distintos cogs.\\

La solución pasa por crear un cog independiente en el que se almacenen las variables globales como atributos a los que solo se pueda acceder a través de unos métodos getters y setters. Este cog sería el primero en crearse y se pasaría como parámetro en la creación de los sucesivos.\\

Las llamadas a los mencionados getters and setters podrían hacerse usando una llamada asíncrona e inmediatamente haciendo un $await$ de esta. Este $await$ sería necesario tanto para lecturas (como se vio en la discusión de los Future (!!!!!)) como en las escrituras, puesto que un mismo cog podría intentar hacer dos escrituras seguidas sobre la misma variable y el orden de ellas debe mantenerse.\\

Usando esta idea el último ejemplo se podría traducir al siguiente código ABS:

\lstinputlisting{code.2.3.txt}

Analicemos por qué esto funciona. La ejecución de $f$ y del resto de la función $main$ se ejecutan ahora en paralelo. Ambas funciones tienen acceso al objeto $globalval$ por ser un parámetro de clase. Cuando $f$ o $main$ realizan una escritura sobre $var1$ llaman al método setter correspondiente. Puede ocurrir que uno de los dos cogs realice la llamada y obtenga el valor de retorno antes que el otro llame al método o que los dos llamen a la vez (relativamente) y hagan un await.\\

En este segundo caso hay que analizar que no haya problemas de deadlock. Los métodos del cog $globalval$ no realizan ninguna llamada a ninguna otra función. Solamente cogen el valor solicitado y lo devuelven, en el caso de las lecturas, o escriben un valor en un atributo, en el caso de las escrituras. Cuando dos o más llamadas a un método ocurren a la vez estas se introducen en la cola de mensajes del cog y se gestionan arbitrariamente. De este modo se garantiza que las llamadas a $globalval$ terminan en algún momento y la arbitrariedad en la gestión de las llamadas garantizan la existencia de todos los interleavings posibles sobre la variable $var1$.\\

Del mismo modo se procede con la variable $var2$. Y gracias al $await$ dentro de un mismo cog (como en la función $f$) se garantiza que no hay un desorden entre la asignación a $var1$ y la asignación a $var2$. De forma análoga esto funcionaría con un ejemplo en el que se hicieran lecturas.\\

Con esto hemos conseguido solucionar el problema de las variables globales y a la vez hemos obtenido una concurrencia de grano fino entre las distintas hebras de CABS al separar las lecturas de la escrituras en dos llamadas asíncronas distintas.

\section{Arrays locales y globales}
