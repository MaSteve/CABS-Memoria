\chapter{ABS: sintaxis y semántica}

ABS\footnote{\url{http://abs-models.org/}} es un lenguaje de modelado para sistemas distribuidos con orientación a objetos y concurrencia basada en actores, donde los objetos de una clase ejecutan sus métodos en función de los mensajes recibidos por otros actores.\\

Sobre este lenguaje se han construido una serie de herramientas para el análisis de programas concurrentes que estamos interesados en mantener para nuestro lenguaje CABS, mediante una traducción de código.\\

Antes de poder continuar con el trabajo de traducción, es necesario hacer una introducción de las construcciones básicas de ABS.

\section{Sintaxis}

Como muestra su manual\footnote{\url{http://abs-models.org/documentation/manual/}}, ABS es un lenguaje que permite el uso de una amplia variedad de construcciones que van desde la definición de tipos de datos algebraicos hasta la definición de funciones dentro de los métodos de una clase.\\

De todas estas construcciones, para este trabajo emplearemos principalmente la definición de clases e interfaces del lenguaje.\\

La sintaxis para la declaración de interfaces en ABS es

\lstinputlisting{code.1.1.txt}

donde las signaturas de los métodos son de la forma

\lstinputlisting{code.1.2.txt}

donde los argumentos son una lista separada por comas de cero o más nombres de variables precedidos de su tipo (al estilo de Java).\\

De los tipos predefinidos en ABS solo usaremos los enteros ($Int$) y los booleanos ($Bool$). También emplearemos el tipo genérico List para la construcción de Arrays.\\

La declaración de clases en ABS es similar a la de Java con la peculiaridad de que todas las clases definidas por el usuario deben implementar una interfaz. La sintaxis para la definición de clases en ABS es

\lstinputlisting{code.1.3.txt}

La declaración de atributos de una clase es idéntica a la de Java, con la ausencia de las palabras reservadas $private$, $public$ o $protected$. La visibilidad de todos los atributos es privada. Del mismo modo, las implementaciones de métodos siguen el mismo estilo. Un método es por tanto público si está definido en la interfaz que implementa la clase, en caso contrario es privado.\\

Una característica especial de las clases de ABS es la posibilidad de definirlas con una lista de argumentos accesibles desde cualquier punto de la clase. Esta propiedad será ampliamente explotada con posterioridad para la implementación del concepto de variable global.\\

Por último nos queda discutir las llamadas a métodos de una clase. En ABS, las llamadas a métodos son un paso de mensaje, es decir, cuando un objeto llama a un método de otro objeto o de sí mismo, dicha llamada se mete en una cola a la espera de poder ser ejecutada por el objeto receptor. Un objeto (o actor) solo puede ejecutar un mensaje a la vez.\\

De los operadores de llamada a métodos de ABS, nosotros solo nos preocuparemos del operador `$!$', cuya sintaxis es

\lstinputlisting{code.1.4.txt}

La idea de esta llamada es mandar un mensaje al objeto $o$ para que ejecute su método $f$. Esta llamada devuelve un tipo futuro. El tipo futuro permite entre otras cosas saber cuando se ha concluido la ejecución del mensaje y en este caso recuperar el valor de rotorno del método.\\

Cuando queramos realizar una llamada síncrona, es decir, una llamada para la que no deseamos continuar la ejecución de un mensaje antes de conocer el valor de retorno del método llamado, emplearemos un $await$. La idea del await es paralizar la ejecución del mensaje actual, permitiendo a otros mensajes del objeto ser ejecutados, y esperar a que la variable futura de la llamada tenga un valor, es decir, que la llamada haya concluido. La construcción $await$ tiene la siguiente sintaxis

\lstinputlisting{code.1.5.txt}

y su tipo de retorno es el mismo que el del método de la llamada.\\

El resto de la sintaxis de ABS empleada en este trabajos se reduce al uso de los $if/else$ y del $while$, así como de la declaración de variables locales en los métodos y de asignaciones a varibles de los resultados de la evaluación de expresiones aritmético-lógicas. La sintaxis referente a estas construcciones del lenguaje son prácticamente similares a las de Java.\\

El concepto de función $main$ en ABS lo cumple un conjunto de instrucciones ABS escritas entre llaves y situadas al final del archivo del código.

\section{Semántica}

De forma similar al desarrollo expuesto para CABS, en esta sección seguiremos unos pasos similares para definir la semántica del lenguaje ABS.\\

Un estado de un programa en ABS vendrá representado por los objetos creados en ejecución y por la información estática aportada por las clases e interfaces, es decir, el código de la implementación de los métodos, tanto públicos como privados, y la inicialización de los atributos. Por tanto definimos $\State_\ABS$ como tuplas $(\O, \Cl)$ donde el elemento $\O$ será una lista de objetos instanciados y $\Cl$ contendrá la definición de las clases e interfaces. Este último elemento será una lista de definiciones indexada por un identificador de clase. Contendrá elementos de la forma $(\mbox{Inter}, attr_c, met, args_c)$ donde, en orden de izquierda a derecha, se tiene el nombre de interfaz que implementa una clase, los atributos o \emph{fields} de la clase, los métodos que implementa junto con su definición (es decir, código y argumentos) y los argumentos de clase.\\

Un objeto vendrá dado a su vez por un identificador único $o$, su nombre de clase, una cola de mensajes a la que llamaremos $\RT$ (Runtime tasks), un identificador de tarea que indique que mensaje está procesando el objeto en ese instante y la información del estado de los atributos mediante $\attr: \Var \hookrightarrow \V_\ABS$, una función que asigne a un nombre de variable un valor de ABS. Puesto que los identificadores de objetos pueden venir dados por un número en $\Z$, se puede ver que $\V_\ABS$ es en esencia el mismo conjunto de valores que hemos definido para CABS. Por comodidad escogeremos esta opción. Es importante tener esto en cuenta puesto que, si usaramos otro tipo de identificador para los objetos, sería necesario definir el conjunto de valores de ABS de otro modo y habría de tenerse en cuenta en el resto de argumentos formales que llevaremos a cabo que los valores en CABS no serían los mismos que en ABS. El papel de $\attr$ será el mismo que tenían los elementos de $\loc$ en la semántica de CABS, permitiendo ahora tener objetos a los que se puede acceder usando un nombre de variable de $\Var$.\\

El concepto de tarea recoge la información del ambito local $loc: \Var \hookrightarrow \V_\ABS$, el código del método $S \in \Stm_\ABS$ y un identificador único de tarea. Con esto quedan definidos los elementos $\RT$. No hay que confundir las variables de ámbito local $loc$, declaradas dentro de un método, con los atributos del objeto $\attr$ que son un añadido al contar con orientación a objetos.\\

Con estas definiciones básicas procedemos a dar una definción formal de la semántica de ABS.
\begin{landscape}

\subsection{Reglas de derivación en ABS}

  \begin{prooftree*}
      \hypo {is\_local(x)}
      \hypo {is\_Int(x)}
    \Infer[left label=$\assa$]2{(\O \leadsto (id, c, \RT \leadsto (loc, x = a\; S, t), t, \attr), \Cl) \stns (\O \leadsto (id, c, \RT \leadsto (loc \ass{x}{\A \eval{a}_{loc, \attr}}, S, t), t, \attr), \Cl)}
  \end{prooftree*}

  \begin{prooftree*}
      \hypo {is\_attr(x)}
      \hypo {is\_Int(x)}
    \Infer[left label=$\assb$]2{(\O \leadsto (id, c, \RT \leadsto (loc, x = a\; S, t), t, \attr), \Cl) \stns (\O \leadsto (id, c, \RT \leadsto (loc, S, t), t, \attr \ass{x}{\A \eval{a}_{loc, \attr}}), \Cl)}
  \end{prooftree*}

  \begin{prooftree*}
    \Infer[left label=$\decla$]0{(\O \leadsto (id, c, \RT \leadsto (loc, Int\, x = a\; S, t), t, \attr), \Cl) \stns (\O \leadsto (id, c, \RT \leadsto (loc \ass{x}{\A \eval{a}_{loc, \attr}}, S, t), t, \attr), \Cl)}
  \end{prooftree*}

  \begin{prooftree*}
      \hypo {is\_local(x)}
      \hypo {is\_Bool(x)}
    \Infer[left label=$\assc$]2{(\O \leadsto (id, c, \RT \leadsto (loc, x = b\; S, t), t, \attr), \Cl) \stns (\O \leadsto (id, c, \RT \leadsto (loc \ass{x}{\Bb \eval{b}_{loc, \attr}}, S, t), t, \attr), \Cl)}
  \end{prooftree*}

  \begin{prooftree*}
      \hypo {is\_attr(x)}
      \hypo {is\_Bool(x)}
    \Infer[left label=$\assd$]2{(\O \leadsto (id, c, \RT \leadsto (loc, x = b\; S, t), t, \attr), \Cl) \stns (\O \leadsto (id, c, \RT \leadsto (loc, S, t), t, \attr \ass{x}{\Bb \eval{b}_{loc, \attr}}), \Cl)}
  \end{prooftree*}

  \begin{prooftree*}
    \Infer[left label=$\declb$]0{(\O \leadsto (id, c, \RT \leadsto (loc, Bool\, x = b\; S, t), t, \attr), \Cl) \stns (\O \leadsto (id, c, \RT \leadsto (loc \ass{x}{\Bb \eval{b}_{loc, \attr}}, S, t), t, \attr), \Cl)}
  \end{prooftree*}

  % Resto de instrucciones

  \begin{prooftree*}
      \hypo {\Bb \eval{b}_{loc, \attr} = \True}
    \Infer[left label=$\aiftabs$]1{(\O \leadsto (id, c, \RT \leadsto (loc, \If{b}{S_1}{S_2} S, t), t, \attr), \Cl) \stns (\O \leadsto (id, c, \RT \leadsto (loc, S_1 S, t), t, \attr), \Cl)}
  \end{prooftree*}

  \begin{prooftree*}
      \hypo {\Bb \eval{b}_{loc, \attr} = \False}
      \Infer[left label=$\aiffabs$]1{(\O \leadsto (id, c, \RT \leadsto (loc, \If{b}{S_1}{S_2} S, t), t, \attr), \Cl) \stns (\O \leadsto (id, c, \RT \leadsto (loc, S_2 S, t), t, \attr), \Cl)}
  \end{prooftree*}

  \begin{prooftree*}
    \Infer[left label=$\wabs$]0{(\O \leadsto (id, c, \RT \leadsto (loc, \while{b}{S_1} S, t), t, \attr), \Cl) \stns (\O \leadsto (id, c, \RT \leadsto (loc, \Ifp{b}{S_1 \while{b}{S_1}} S, t), t, \attr), \Cl)}
  \end{prooftree*}

  \begin{prooftree*}
      \hypo{\Cl\eval{c'} = (\mbox{Inter}, attr_c, met, args_c)}
      \hypo{check\_args(args_c, args)}
    \Infer[left label=$\oabs$]2{(\O \leadsto (id, c, \RT \leadsto (loc,  \mbox{Inter } inter = new\, c'(args)\; S, t), t, \attr), \Cl) \stns (\O:o \leadsto (id, c, \RT \leadsto (loc \ass{inter}{id'}, S, t), t, \attr), \Cl)}
  \end{prooftree*} donde $o = (id', c', [], \perp, \attr')$ con $id'$ un nuevo identificador de objeto no utilizado y $\attr' = attr_c \ass{args_c}{\E\eval{args}_{loc, \attr}}$ los atributos del nuevo objeto creado.

  \begin{prooftree*}
      \hypo{loc \cup attr \eval{inter} = id'}
      \hypo{\Cl\eval{c'} = (\mbox{Inter}, attr_c, met, args_c)}
      \hypo{contains(met, m)}
      \hypo{check\_args(args_m, args)}
    \Infer[left label=$\tskabsa$]4{(\O \leadsto  (id', c', \RT', t', \attr') (id, c, \RT \leadsto (loc, inter!m(args)\; S, t), t, \attr), \Cl) \stns (\O \leadsto (id', c', \RT':tsk, t', \attr') (id, c, \RT \leadsto (loc, S, t), t, \attr), \Cl)}
  \end{prooftree*} donde $tsk = (loc', S', t'')$ con $t''$ un identificador de tarea nuevo y $loc' = \nil\ass{args_m}{args}$ y $met\eval{m} = (S', args_m)$

   \begin{prooftree*}
      \hypo{loc \cup attr \eval{int} = id'}
      \hypo{\Cl\eval{c'} = (\mbox{Inter}, attr_c, met, args_c)}
      \hypo{contains(met, m)}
      \hypo{check\_args(args_m, args)}
    \Infer[left label=$\tskabsb$]4{(\O \leadsto  (id', c', \RT', t', \attr') (id, c, \RT \leadsto (loc, Int\,x = await\,int!m(args)\; S, t), t, \attr), \Cl) \stns (\O \leadsto o' (id, c, \RT \leadsto (loc, Int\,x = await\,t'\; S, t), t, \attr), \Cl)}
   \end{prooftree*} donde $o' = (id', c', \RT':tsk, t', \attr')$, $tsk = (loc', S', t'')$ con $t''$ un identificador de tarea nuevo y $loc' = \nil\ass{args_m}{args}$ y $met\eval{m} = (S', args_m)$

   \begin{prooftree*}
      \hypo{loc \cup attr \eval{int} = id'}
      \hypo{\Cl\eval{c'} = (\mbox{Inter}, attr_c, met, args_c)}
      \hypo{contains(met, m)}
      \hypo{check\_args(args_m, args)}
    \Infer[left label=$\tskabsbb$]4{(\O \leadsto  (id', c', \RT', t', \attr') (id, c, \RT \leadsto (loc, await\,int!m(args)\; S, t), t, \attr), \Cl) \stns (\O \leadsto o' (id, c, \RT \leadsto (loc, await\,t'\; S, t), t, \attr), \Cl)}
   \end{prooftree*} donde $o' = (id', c', \RT':tsk, t', \attr')$, $tsk = (loc', S', t'')$ con $t''$ un identificador de tarea nuevo y $loc' = \nil\ass{args_m}{args}$ y $met\eval{m} = (S', args_m)$

  \begin{prooftree*}
      \hypo{tsk = (loc', \varepsilon(\nu), t'')}
    \Infer[left label=$\retabs$]1{(\O \leadsto  (id', c', \RT' \leadsto tsk, t', \attr') (id, c, \RT \leadsto (loc, Int\,x = await\,t'\; S, t), t, \attr), \Cl) \stns (\O \leadsto o' (id, c, \RT \leadsto (loc\ass{x}{\nu}, S, t), t, \attr), \Cl)}
  \end{prooftree*} donde $o' = (id', c', \RT', t', \attr')$

  \begin{prooftree*}
      \hypo{tsk = (loc', \varepsilon(\nu), t'')}
    \Infer[left label=$\retabsb$]1{(\O \leadsto  (id', c', \RT' \leadsto tsk, t', \attr') (id, c, \RT \leadsto (loc, await\,t'\; S, t), t, \attr), \Cl) \stns (\O \leadsto o' (id, c, \RT \leadsto (loc, S, t), t, \attr), \Cl)}
  \end{prooftree*} donde $o' = (id', c', \RT', t', \attr')$

  \begin{prooftree*}
    \hypo{tsk = (loc', S, t'')}
    \hypo{S \neq \varepsilon(\nu)}
    \Infer[left label=$\waitabs$]2{(\O \leadsto  (id', c', \RT' \leadsto tsk, t', \attr') (id, c, \RT \leadsto (loc, Int\,x = await\,t'\; S, t), t, \attr), \Cl) \stns (\O \leadsto o' (id, c, \RT \leadsto (loc, Int\,x = await\,t'\; S, t), \perp, \attr), \Cl)}
  \end{prooftree*} donde $o' = (id', c', \RT', t', \attr')$

  \begin{prooftree*}
    \Infer[left label=$\selabs$]0{(\O \leadsto (id, c, \RT \leadsto (loc, S, t), \perp, \attr), \Cl) \stns (\O \leadsto (id, c, \RT \leadsto (loc, S, t), t, \attr), \Cl)}
  \end{prooftree*}

  % Return: igual conviene tener algo más en la task que indique que está activa o muerta.
  \begin{prooftree*}
    \Infer[left label=$\eabs$]0{(\O \leadsto (id, c, \RT \leadsto (loc, \return a\;S, t), t, \attr), \Cl) \stns (\O \leadsto (id, c, \RT \leadsto (loc, \varepsilon(\A \eval{a}_{loc, \attr}), t), \perp, \attr), \Cl)}
  \end{prooftree*}

  \begin{prooftree*}
    \Infer[left label=$\dselabs$]0{(\O \leadsto (id, c, \RT \leadsto (loc, \varepsilon(\nu), t), t, \attr), \Cl) \stns (\O \leadsto (id, c, \RT \leadsto (loc, \varepsilon(\nu), t), \perp, \attr), \Cl)}
  \end{prooftree*}

\end{landscape}

\section{Extensión semántica y sintáctica}

Además de las definiciones anteriores asociadas al lenguaje ABS, puede resultar interesante añadir una serie de estructuras auxiliares a modo de azúcar sintáctico que puedan facilitarnos el camino en los siguientes capítulos de este trabajo. Es por ello por lo que hemos introducido en la semántica la notación $await\,t$ para indicar que esperamos a que termine la tarea con identificador $t$ y que recuperamos su valor de retorno. Este tipo de construcción y otras más pueden no existir en el lenguaje original ABS, pero nosotros haremos uso de ellas como mera notación para expresar el significado del resto de construcciones sí presentes en el lenguaje.\\

\section{Ejemplos}

Sea el código en ABS
\lstinputlisting{code.1.6.txt}

Tras ejecutar el código de inicialización, en el que creamos el objeto $globaval$ e instanciamos un objeto que contiene la implementación de $main$ a la que llamamos con la instrucción $await\, prog!main()\;$, nos encontramos en un estado
$$
((0, GlobalVariables, [], \perp, \nil\ass{var}{0}):(1, Impmain, (\nil, S, 0), \perp, \nil\ass{globalval}{0}), \Cl)
$$

donde $S$ es el código de $main$. La taréa 0 del objeto 1 puede entrar a ejecutarse con la regla $\selabs$ y podemos suponer que se ejecuta por completo, dando lugar por el camino a la creación de dos objetos de la clase $Impf$, lo que nos lleva al estado
\begin{multline*}
  ((0, GlobalVariables, [], \perp, \nil\ass{var}{0}):\\
  (1, Impmain, (\nil, \epsilon(0), 0), \perp, \nil\ass{globalval}{0}):\\
  (2, Impf, (\nil\ass{value}{1}, S', 1), \perp, \nil\ass{globalval}{0}):\\
  (3, Impf ,(\nil\ass{value}{2}, S', 2), \perp, \nil\ass{globalval}{0}), \Cl)
\end{multline*}
donde $S' = await\, globalval!setvar(value)\;$.\\

En cualquier momento pueden ser seleccionadas las tareas 1 y 2 en sus respectivos objetos, creando dos tareas en el objeto 0, asociadas a cada una de las asignaciones sobre la variable global $var$. Llegando por ejemplo al estado
\begin{multline*}
  ((0, GlobalVariables, (\nil\ass{val}{1}, S'', 3):(\nil\ass{val}{2}, S'',4), \perp, \nil\ass{var}{0}):\\
  (1, Impmain, (\nil, \epsilon(0), 0), \perp, \nil\ass{globalval}{0}):\\
  (2, Impf, (\nil\ass{value}{1}, await\,3, 1), \perp, \nil\ass{globalval}{0}):\\
  (3, Impf ,(\nil\ass{value}{2}, await\,4, 2), \perp, \nil\ass{globalval}{0}), \Cl)
\end{multline*}
donde $S''$ es el código que asigna $val$ al atributo $var$ de la clase.\\

Ahora la regla $\selabs$ puede escoger alguna de las dos tareas del objeto 0. Es en este punto donde la selección de tarea dará lugar a los dos entrelazamientos posibles del programa. Suponiendo que primero se ejecute la tarea 3 y posteriormente la tarea 4 se llega a
\begin{multline*}
  ((0, GlobalVariables, (\nil\ass{val}{1}, \varepsilon, 3):(\nil\ass{val}{2}, \varepsilon, 4), \perp, \nil\ass{var}{2}):\\
  (1, Impmain, (\nil, \epsilon(0), 0), \perp, \nil\ass{globalval}{0}):\\
  (2, Impf, (\nil\ass{value}{1}, await\,3, 1), \perp, \nil\ass{globalval}{0}):\\
  (3, Impf ,(\nil\ass{value}{2}, await\,4, 2), \perp, \nil\ass{globalval}{0}), \Cl)
\end{multline*}

En este momento los $await$ de las tareas 1 y 2 pueden concluir y llegamos al estado final
\begin{multline*}
  ((0, GlobalVariables, (\nil\ass{val}{1}, \varepsilon, 3):(\nil\ass{val}{2}, \varepsilon, 4), \perp, \nil\ass{var}{2}):\\
  (1, Impmain, (\nil, \epsilon(0), 0), \perp, \nil\ass{globalval}{0}):\\
  (2, Impf, (\nil\ass{value}{1}, \varepsilon, 1), \perp, \nil\ass{globalval}{0}):\\
  (3, Impf ,(\nil\ass{value}{2}, \varepsilon, 2), \perp, \nil\ass{globalval}{0}), \Cl)
\end{multline*}
