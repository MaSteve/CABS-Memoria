\chapter{Corrección}

A lo largo de este capítulo usaremos las definiciones de las semánticas previamente expuestas para comprobar que la traducción propuesta de CABS a ABS es correcta. En otras palabras, durante las próximas secciones, demostraremos que dado un código en CABS y partiendo de un estado inicial podemos ``ejecutar'' una serie de pasos hasta llegar a un nuevo estado que tiene un homólogo en ABS y que resulta de ejecutar la traducción desde un estado equivalente al inicial en CABS (y viceversa).\\

Este procedimiento es conocido como bisimulación y nos permitirá hacer afirmaciones tan fuertes como las que buscamos, es decir, que si tenemos un código en CABS y dada su traducción sabemos, usando la herramientas ya desarrolladas para ABS, que cumple una cierta propiedad partiendo de un estado, sabremos entonces que el mismo resultado se sostiene para el código en CABS.

\section{Equivalencia entre estados}

Sea $(\G, \F, \RP)$ un estado en CABS y $(\O, \Cl)$ un estado en ABS. Diremos que estos estados son equivalentes si:
\begin{itemize}
\item dado $(id, c, \RT, t, \attr) \in \O$ el objeto correspondiente a la clase GlobalVariables, tenemos que las funciones $\G$ y $\attr$ son la misma.
\item para todo nombre de función $\func$ con $\F(\func) = (t, S, args, a)$, tenemos que $\Cl\eval{\text{Imp}\func} = (\text{Int}\func, \nil, met, \args_c)$ donde $met\eval{\func}$ contiene la traducción de los argumentos $args$ y del código de la función y los argumentos de clase $\args_c$ contienen solo una variable con el tipo de la interfaz de las variables globales.
\item cada elemento en $\RP$ se corresponde con un conjunto de tareas en los objetos de $\O$, en concreto, cada entorno de variables apilado se corresponde con los atributos (obviando variables auxiliares) de una tarea cuyo código traducido se corresponde con un segmento de instrucciones del código del proceso. Lo que se quiere decir con esto es que, mientras que las llamadas a función en CABS se corresponden con el apilamiento de un nuevo entorno de variables y la concatenación del código de la función con el código actual del proceso, en ABS se crea una nueva tarea a la que se espera para obtener el valor de retorno.
\item en caso de existir un elemento $(local:s, x = e\; S) \in \RP$ donde $e$ tiene elementos en $\V$ ya calculados, se tiene que en la tarea del estado equivalente se han ejecutado ya las asignaciones a las variables auxiliares correspondientes con los valores ya calculados y, por tanto, quedan por calcular las asignaciones a variables auxiliares restantes en la expresión. En otras palabras, cada paso de evaluación de una expresión será una asignación en la traducción en ABS. Esto es válido tanto para enteros como para booleanos.
\end{itemize}
\section{Corrección de las expresiones aritmético-lógicas}
Una de las peculiaridades que tiene CABS, es permitir en sus expresiones una notación para realizar llamadas a función con retorno de valor. Como ya notamos en el capítulo anterior, esto hace que la traducción se apoye en el uso de unas variables auxiliares que permiten dividir el cómputo de una expresión en el cómputo de las distintas subexpresiones que la componen.\\

Cada expresión utilizada en un programa CABS se corresponde de forma unívoca con una variable auxiliar en ABS que almacenará el valor de la expresión cuando este esté disponible. De este modo, en todo momento podemos hacer uso de esta información, aunque, por simplificar la notación, no esté presente en el estado del programa.\\

Hecha esta introducción, procedemos a comprobar en primer lugar que las asignaciones hechas en CABS se corresponden con las de la traducción de estas a ABS, llegando de estados equivalentes a estados equivalentes en un solo paso.

\subsection{Variables locales}

\subsubsection{Caso $x = \nu$ con $\nu \in \V$}
Sean $(\G, \F, \RP \leadsto (local:s,x = \nu\;S))$ y $(\O \leadsto (id, c, \RT \leadsto (loc, x = aux\_var\_k\; S_\ABS, t), t, \attr), \Cl)$ estados equivalentes, donde la variable auxiliar $k$-esima es la variable asignada a la expresión $\nu$ de forma que $loc(aux\_var\_k) = \nu$.\\

Tenemos que del estado $(\G, \F, \RP \leadsto (local:s,x = \nu\;S))$, aplicando la regla $\assvc$, llegamos al estado $(\G, \F, \RP \leadsto (local\ass{x}{\nu}:s, S))$ en CABS. Por otro lado, del estado equivalente en ABS $(\O \leadsto (id, c, \RT \leadsto (loc, x = aux\_var\_k\; S, t), t, \attr), \Cl)$, aplicando la regla de asignación local correspondiente, llegamos al estado $(\O \leadsto (id, c, \RT \leadsto (loc \ass{x}{\A \eval{aux\_var\_k}_{loc, \attr}}, S_\ABS, t), t, \attr), \Cl)$ y del hecho de que dicha variable auxiliar tenga el valor de $\nu$ llegamos inmediatamente a un estado equivalente.

\subsubsection{Caso $x = n$}
Sea $(\G, \F, \RP \leadsto (local:s,x = n\;S))$ y sea $(\O \leadsto (id, c, \RT \leadsto (loc, Int\,aux\_var\_k = n\; x = aux\_var\_k\; S_\ABS, t), t, \attr), \Cl)$ su estado equivalente, donde la variable auxiliar $k$-esima es la variable asignada a la expresión $n$.\\

En este caso, solo podemos aplicar para el proceso actual la regla $\assac$ que delega la acción en las reglas para expresiones aritmético-lógicas. La regla aplicable en esta situación es $\numa$ con lo que llegamos al estado $(\G, \F, \RP \leadsto (local:s,x = \N\eval{n}\;S))$.\\

En el lado de ABS aplicamos la regla de asignación sobre la variable auxiliar llegando al estado $(\O \leadsto (id, c, \RT \leadsto (loc \ass{aux\_var\_k}{\A \eval{n}_{loc, \attr}}, x = aux\_var\_k\;S_\ABS, t), t, \attr), \Cl)$ preservandose la equivalencia entre estados al asignar el valor de la expresión a su variable auxiliar.

\subsubsection{Caso $x = f(\cdots)$ (simplificación sin argumentos!!!)}
Sean los estados equivalentes $(\G, \F, \RP \leadsto (local:s,x = f()\;S))$ y $(\O \leadsto (id, c, \RT \leadsto (loc, \mbox{Intf }  aux\_var\_(k-1) = new\, \mbox{Impf}(globalval) \; Int\, aux\_var\_k = await\,auxi\_var\_(k-1)!f()\; x = aux\_var\_k\; S_\ABS, t), t, \attr), \Cl)$ donde la variable auxiliar $k$-esima es la variable asignada a la expresión $f()$.\\

La semántica de CABS nos permite usar la regla $\assac$ con la regla $\calla$ resultando en el estado $(\G, \F, \RP \leadsto (\nil:local:s, S_F\,x = \Unstack{a}\;S))$ donde $a$ es la expresión de retorno de $f$.\\

Por otro lado, en la semántica de ABS, podemos aplicar primero la regla de creacción de objetos y a continuación la regla de llamada síncrona, resultando el estado
\begin{multline*}
  (\O \leadsto (id', c', (\nil, S_\ABS^F, t'), \perp, \attr')\\
  (id, c, \RT \leadsto (loc, Int\, aux\_var\_k = await\,t'\; x = aux\_var\_k\; S_\ABS, t), t, \attr), \Cl)
\end{multline*}

Llegados a este punto tenemos un estado que es equivalente al resultante en la semántica de CABS puesto que el concepto de apilar un nuevo ámbito de variables en ABS se ve reflejado creando un objeto nuevo que tiene como única tarea asignada el cómputo del código traducido de la función $f$. El objeto original se ve pausado por el $await$ hasta que no termine de ejecutarse el código de $f$, imitando el concepto del código concatenado en CABS que no deja ejecutar resto de instrucciones hasta no terminar con las de la función.\\

Otro detalle a tener en cuenta es el indicador de tarea actual. Tras dar los dos paso anteriores en la semántica de ABS se nos lleva a tener como tarea asignada en el nuevo objeto a $\perp$. Evidentemente, antes de poderse ejecutar la primera instrucción de $f$ es necesario que se ejecute la regla de selección de tarea. El hecho de que ningún objeto de nuestra traducción vaya a ejecutar más de una tarea (a excepción del objeto de variables globales del que hablaremos más adelante) nos permite extender la idea de equivalencia a este estado actual, puesto que consideramos que aplicar una regla de selección no afecta en esencia a la configuración actual en ABS ya que solo lo hacen los ``movimientos'' en el código.\\

De forma similar, el hecho de tener que dar dos pasos tampoco afecta, pese a existir una concurrencia en la semántica. El motivo es que entre ambos pasos el objeto creado solo es referenciado por una variable auxiliar de modo que se garantiza que no puede ser modificado desde ningún otro punto de la ejecución. De este modo podemos volver a extender el concepto de equivalencia permitiendo que el paso intermedio de creación del objeto (omitido en este texto) pueda ser también considerado un estado que mantiene la equivalencia con $(\G, \F, \RP \leadsto (\nil:local:s, S_F\,x = \Unstack{a}\;S))$. Con esto se quiere decir que, al igual que las reglas de selección de tarea, la creación de objetos no altera tampoco el estado intrínseco del programa.

\subsubsection{Caso $x = \Unstack{\nu}$ con $\nu \in \V$}
Sean los estados equivalentes $(\G, \F, \RP \leadsto (local':local:s,x = \Unstack{\nu}\;S))$ y
\begin{multline*}
  (\O \leadsto (id', c', (loc', \return aux\_var\_k', t'), t', \attr')\\
  (id, c, \RT \leadsto (loc, Int\, aux\_var\_k = await\,t'\; x = aux\_var\_k\; S_\ABS, t), t, \attr), \Cl)
\end{multline*}
donde la variable auxiliar $k'$-esima es la variable asignada a la expresión $\nu$, que tiene su valor asignado en $loc'$, y la $k$-esima al unstack.\\

En el caso de CABS, aplicar la regla $\assac$ con la regla $\unab$ para la expresión nos lleva al estado $(\G, \F, \RP \leadsto (local:s, x = \nu \; S))$.\\

En el caso de ABS, podemos aplicar la regla de retorno que nos lleva a un estado intermedio
\begin{multline*}
  (\O \leadsto (id', c', (loc', \varepsilon(\A \eval{aux\_var\_k'}_{loc, \attr}), t'), t', \attr')\\
  (id, c, \RT \leadsto (loc, Int\, aux\_var\_k = await\,t'\; x = aux\_var\_k\; S_\ABS, t), t, \attr), \Cl)
\end{multline*}

Desde este estado se puede aplicar la regla del $await$ que permite recoger el valor de $\nu$ y asignarselo, en este caso, a la variable $aux\_var\_k$, llegando al estado
\begin{multline*}
  (\O \leadsto (id', c', [], \perp, \attr')\\
  (id, c, \RT \leadsto (loc\ass{aux\_var\_k}{\nu}, x = aux\_var\_k\; S_\ABS, t), t, \attr), \Cl)
\end{multline*}

que mantiene la equivalencia.\\

De nuevo se nos presenta la discusión de los pasos múltiples. El mismo argumento que hemos presentado anteriormente es aplicable a esta situación.\\

Cabe destacar que el concepto de desapilar un entorno en CABS queda reflejado en el hecho de que el objeto con el ámbito superior no vuelve a ser usado y queda vacío de tareas.

\subsubsection{Caso $x = a_1 \odot a_2$}

Antes de discutir este caso conviene ver que en CABS el cómputo de $a_1 \odot a_2$ precisa del cómputo de $a_1$, de modo que el argumento que vamos a seguir para el caso compuesto es asumir que todo va bien si nos encontraramos con la expresión $a_1$ y que por tanto la asunción se puede emplear para la expresión $a_1 \odot a_2$.\\

Otro detalle importante a tener en cuenta es que no se empieza a procesar la expresión $a_2$ hasta que no hemos completado el cómputo de $a_1$, como se refleja en las reglas de la semántica de CABS. Esto se reflejará también en ABS ya que las asignaciones de la expresión $a_2$ vienen precedidas por las de la expresión $a_1$.\\

Dicho esto, sean los estados equivalentes $(\G, \F, \RP \leadsto (local:s,x = a_1 \odot a_2\;S))$ y
\begin{multline*}
  (\O \leadsto (id, c, \RT \leadsto (loc, c_1 c_2 \,Int\,aux\_var\_k\ = aux\_var\_k_1 \odot aux\_var\_k_2 \;\\
  x = aux\_var\_k\; S_\ABS, t), t, \attr), \Cl)
\end{multline*}
donde $c_1$ es el código asociado a $a_1$, $c_2$ el asociado a $a_2$ y $aux\_var\_k_1$, $aux\_var\_k_2$ y $aux\_var\_k$ las variables asociadas a cada una de las expresiones aritméticas en juego.\\

En concreto tenemos que la última instrucción de $c_1$, por el modo en que hemos definido la traducción, se corresponde con una asignación a su variable auxiliar. Por el modo en que están definidas las expresiones aritméticas, podemos aplicar inducción estructural con lo que, por hipotesis de inducción tenemos que los estados equivalentes $(\G, \F, \RP \leadsto (local:s,x = a_1\;S))$ y $(\O \leadsto (id, c, \RT \leadsto (loc, c_1\,x = aux\_var\_k_1\; S_\ABS, t), t, \attr), \Cl)$ llegan en un solo paso a los estados equivalentes $(\G, \F, \RP \leadsto (local:s,x = a_1'\;S))$ y $(\O \leadsto (id, c, \RT \leadsto (loc, c_1'\,x = aux\_var\_k_1\; S_\ABS, t), t, \attr), \Cl)$.\\

Teniendo en cuenta esto y el lema (escribir lema!!! La idea es que si tienes como código $S_1$ y un estado $s$ y se ejecuta un paso para llegar a $S_1'$ y $s'$ tenemos que de $S_1 S_2$ y $s$ pasamos a $S_1' S_2$ y $s'$), llegamos a que de los estados originales llegamos en un paso a los estados $(\G, \F, \RP \leadsto (local:s,x = a_1' \odot a_2\;S))$ y
\begin{multline*}
  (\O \leadsto (id, c, \RT \leadsto (loc, c_1' c_2 \,Int\,aux\_var\_k\ = aux\_var\_k_1 \odot aux\_var\_k_2 \;\\
  x = aux\_var\_k\; S_\ABS, t), t, \attr), \Cl)
\end{multline*}
que son equivalentes.

\subsubsection{Casos $x = \nu \odot a_2$ y $x = \nu \odot \mu$ con $\nu, \mu \in \Z$}

La situación del primer caso es completamente análoga a la del caso anterior con la diferencia de que la hipótesis de inducción es aplicada sobre la expresión $a_2$.\\

Para el segundo caso tenemos los estados equivalentes $(\G, \F, \RP \leadsto (local:s,x = \nu \odot \mu\;S))$ y $(\O \leadsto (id, c, \RT \leadsto (loc, Int\,aux\_var\_k =  aux\_var\_k_1 \odot  aux\_var\_k_2\; x = aux\_var\_k\; S_\ABS, t), t, \attr), \Cl)$ que de forma análoga al caso $x = n$ llegan en un paso a $(\G, \F, \RP \leadsto (local:s,x = \nu \odot_\N \mu\;S))$ y
\begin{multline*}
  (\O \leadsto (id, c, \RT \leadsto (loc \ass{aux\_var\_k}{\A \eval{aux\_var\_k_1 \odot  aux\_var\_k_2}_{loc, \attr}},\\
  x = aux\_var\_k\;S_\ABS, t), t, \attr), \Cl)
\end{multline*}
que son equivalentes.

\subsubsection{Casos $x = f(a_1, \dots, a_n)$ y $x = \Unstack{a}$}
Podemos considerar la llamada a una función con argumentos o la instrucción unstack como operadores aritméticos ($n$-ario el primero, unario el segundo) que del mismo modo que los anteriores operadores binarios esperan al computo de las subexpresiones para poder ejecutarse.\\

El proceder con estos es identico en el sentido de aplicar la hipótesis de inducción con las subexpresiones y componer de nuevo con el operador. La llamada final a la función cuando están preparados todos los operandos o la ejecución del unstack están reflejados ya en los casos base presentados al principio de la subsección.

\subsection{Variables globales (TODO!!!)}

\section{Corrección del resto de instrucciones}

\subsubsection{Declaraciones de variable}
Este caso es inmediato puesto que por definición en CABS una declaración inicializa la variable a cero (o false para los booleanos) a nivel semántico y en nuestra traducción forzamos esta situación con una declaración y una asignación. En ambas situaciones, con las reglas $\declic$ o $\declbc$ de CABS y las reglas (!!!) de ABS, llegamos en un paso a estados equivalentes.

\subsubsection{Caso if}
El if vuelve a presentar una situación similar a la que hemos tenido con las expresiones aritméticas y la que hemos omitido para las expresiones booleanas.\\

Tenemos que las dos reglas axiomáticas del if en CABS (la regla para el valor true y el valor false de la condición) son los casos base de un desarrollo inductivo como el que hemos comentado anteriormente y se corresponden con un único paso de ejecución en ABS con la regla del if. La tercera regla de CABS es sobre la que se emplea la hipótesis de inducción y se corresponde con los pasos de ejecución de la expresión booleana.\\

De este modo, es fácil ver que volvemos a tener que la ejecución de un paso en el if en CABS (ya sea en la condición o en la decisión de que rama se toma) se corresponde con un paso de ejecución (obviando pasos no relevantes, como los de cambio de tarea) en ABS.

\subsubsection{Caso while}
El problema que se nos presenta con el while es que su traducción a ABS no preserva la equivalencia de estados inmediatamente al aplicar las reglas de la semántica. De hecho, en CABS el while junto con su condición mutan en un if equivalente resultante de desenrollar el blucle una vez, mientras que, en ABS, la traducción empieza por computar las expresiones boolenas de la condición y cuando tiene el resultado listo se convierte en un if para entonces sí volver a ser un estado equivalente.\\

Hay varias formas de solucionar este problema. Una de ellas pasaría por hacer una modificación en la semántica de CABS, creando una prórroga antes de realizar la conversión al if. Esta opción, aunque complicaría un poco más la semántica del lenguaje, nos permitiría tener de nuevo una equivalencia paso a paso como la que hemos tenido hasta ahora. Las nuevas reglas del while en CABS serían las siguientes:


\begin{prooftree*}
  \hypo{\st{b}{(\G, \F, \RP \leadsto (local:s, \varepsilon))} \stbexp \st{b'}{(\G, \F, \RP \leadsto (s', S_{\B}))}}
  \Infer[left label=$\wacalt$]1{
     \begin{aligned}
       (\G, \F, \RP \leadsto (local:s, \while{b}{S_1}S)) \stns \\
       (\G, \F, \RP \leadsto (s', S_{\B}\,\while{b'}{S_1}S))
     \end{aligned}
  }
\end{prooftree*}\\
\begin{prooftree*}
  \Infer[left label=$\wnuc$]0{
    \begin{aligned}
      (\G, \F, \RP \leadsto (local:s, \while{\nu}{S_1}S)) \stns \\
      (\G, \F, \RP \leadsto (local:s, \Ifp{\nu}{S_1\while{b}{S_1}}\;S))
    \end{aligned}
  }
\end{prooftree*} donde $\nu$ es un valor booleano y $b$ es la expresión booleana original de la condición del while.\\

Una de las primeras asunciones que se tienen en cuenta en estas reglas es la estatismo del código. A la hora de representar las reglas semánticas se ha optado por una semántica de paso corto en la que el código evoluciona indicando lo que queda de ejecución. Este no es el caso real de las implementaciones habituales en las que el código se mantiene inmutable. No obstante, y por motivos de simplicidad, generalmente, se opta por esta aproximación en la que el código muta. Es por ello que en la regla $\wnuc$ tenemos que asumir que podemos recuperar la condición booleana original para poder reescribir el bucle original en el interior del if. Asunción que por otra parte que no es muy exigente.\\

La segunda cuestión que nos debemos plantear es si sustituir la regla del while original por estas dos nuevas reglas no modifica el significado de los programas. La forma de ver que esto no ocurre pasa por usar inducción sobre la longitud de la secuencia de ejecución de la evaluación de la condición desde el estado $(\G, \F, \RP \leadsto (local:s, \while{b}{S_1}S))$, en la que solo se emplean las reglas del while y del if.\\

La idea es que si $b \in \B$ tendríamos en un solo paso la misma transición que llega al estado $(\G, \F, \RP \leadsto (local:s, \Ifp{\nu}{S_1\while{b}{S_1}}\;S))$, luego el caso base estaría resuelto. Para cadenas de longitud $k$ suponemos que se cumple, tras la ejecución de esos $k$ pasos en ambas semánticas, la igualdad de los estados finales y veamos que se sigue para cadenas de $k+1$ pasos.\\

Supongamos que el estado $(\G, \F, \RP \leadsto (local:s, \while{b}{S_1}S))$ necesita $k+1$ pasos en la semántica modificada para ``dar el salto'' al if. Aplicando la regla $\wacalt$ llegamos al estado $(\G, \F, \RP \leadsto (s', S_{\B}\,\while{b'}{S_1}S))$. Por otro lado, en la semántica original, el while se ve como un salto inmediato al estado $(\G, \F, \RP \leadsto (local:s, \Ifp{b}{S_1\;\while{b}{S_1}}S))$ sobre el que ahora se aplica la misma ejecución de la expresión booleana anterior (por ser la misma expresión $b$) y llegamos al estado $(\G, \F, \RP \leadsto (s', S_{\B}\,\Ifp{b'}{S_1\;\while{b}{S_1}}S))$. Ahora, aplicando la hipótesis de inducción, tenemos que ambos estados llegan al mismo estado en $k$ pasos con lo que concatenando con el paso anterior tenemos que la propiedad se mantiene.\\

Una vez tenemos resueltas estas cuestiones, la equivalencia en la bisimulación entre CABS y ABS se sigue de un argumento completamente análogo al del caso del if.\\

\textit{Opción alternativa.}\\

Otro modo de concluir con el caso del while sería crear un azucar sintáctico para el while de ABS que permitiera agrupar en su sintaxis el código de las asignaciones para el cómputo de la condición. De este modo, se podría modificar la regla del while para que actuara sobre esta construcción auxiliar y pudiera realizar el mismo paso que la semántica de CABS original para llegar a un equivalente a la vez. Esta opción sería equivalente a la anterior pero por simplicidad se deja solo comentada.

\subsubsection{Caso thread}
