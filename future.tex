\chapter{Conclusiones y trabajo futuro}

Con este capítulo concluimos este Trabajo de Fin de Grado. A lo largo de estas páginas hemos podido ver cómo definir la sintaxis y la semántica de un lenguaje concurrente, similar a C, para poder posteriormente formalizar una traducción correcta sobre un lenguaje con un paradigma de concurrencia distinto, como ABS y la concurrencia basada en actores, donde existen una serie de herramientas para el análisis de programas concurrentes, demostrando que los resultados extraidos del código traducido pueden extenderse a los que uno obtendría de ejecutar el código origen en CABS. En especial, queda contemplado el uso de SYCO para obtener todos los \emph{interleavings} presentes en un programa en ABS, y ahora en CABS.\\

CABS puede ser de gran utilidad en asignaturas como Programación Concurrente donde la sencillez de su sintaxis permitiría hacer pruebas rápidas sobre algunos problemas de concurrencia donde sea neceario obtener los estados finales de la ejecución de un programa determinado, permitiendo a los alumnos obtener un modo de comprobar sus soluciones y al docente un modo alternativo para impartir sus clases y mostrar ejemplos.\\

El objetivo que se pretendía conseguir con CABS era el de acercar algunas de las herramientas de testing más novedosas a un lenguaje más próximo al de una implementación real en C, en lugar de emplear un lenguaje de modelado como ABS que requeriría una posterior implementación en un lenguaje de programación, siendo Java el más parecido para este lenguaje.\\

El actual CABS dista aún mucho de un lenguaje completo como C, tratándose por el momento de un subconjunto funcional de este. Un futuro trabajo que podría surgir de este consistiría en completar CABS con las construcciones restantes de C, como son los \emph{switch}, los bucles \emph{for} y \emph{do-while}, así como, un mecanismo de entrada y salida que hiciera de CABS un lenguaje más \emph{user friendly}. Del mismo modo, podría ser interesante ampliar los mecanismos de concurrencia implementados para acercarnos más a las funciones de una librería como \emph{pthread} en C, ofreciendo soporte para mecanismos de control, como semáforos o cerrojos, o la posibilidad de esperar la finalización de un hilo para continuar la ejecución del hilo que lo invocó.\\

La aproximación de lenguaje CABS al lenguaje C permitiría el uso de las herramientas implementadas sobre ABS en un lenguaje ampliamente usado en el ámbito de los sistemas operativos, donde es de vital importancia el correcto uso de la concurrencia, añadiendo de este modo una razón más para continuar con el desarrollo de este lenguaje de programación que hemos creado en este trabajo.
